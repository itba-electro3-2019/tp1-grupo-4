\section{Cálculo de REsolución y rango con convención de punto fijo}
El sistema binario es un sistema de numeración en el que los números se representan utilizando unicamente dos cifras. Es fundamental para la lógica computacional, debido a la simpleza y naturalidad con el que puede aplicarse al funcionamiento de una computadora o una máquina digital en general. El código binario permite representar números enteros, racionales e incluso signados. Existen diferentes maneras de representar el signo mediante el sistema binario, pero la más usada es por el complemento a dos del número. Existen además distintas convenciones para representar un número racional en binario. En este ejercicio se usará una de ellas: la convención de números en punto o coma fija. En ella se trabaja con un número fijo de bits y se acuerda dejar una cantidad determinada de bits para trabajar la parte fraccionaria del número. 

En este ejercicio se escribió un programa al cual se le ingresa si el número es signado o no, cuantos bits de parte entera y cuantos bits de parte farccionaria hay. Con esos datos se devuelven dos caracteríticas del número: su resolución, o la cantidad más pequeña representable con la convención ingresada, y su rango, el número más grande representable menos el más pequeño. 

El programa se escribió en Python 3.7 y no se requieren de paquetes externos para su funcionamiento.