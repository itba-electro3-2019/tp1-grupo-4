\documentclass[x11names,border=10pt,tikz]{standalone}
\usepackage{xstring}    % needed for string manipulations
\usepackage{fmtcount}   % needed for some binary printing
\usetikzlibrary{calc,math}
\newif\ifKaddress
\newif\ifKInd
\pgfkeys{
  /K/.is family, /K,
  x bits/.estore in              = \KXvars,     % bits along the horizontal axis
  y bits/.estore in              = \KYvars,     % bits along the vertical axis
  variable names/.estore in      = \KVars,      % comma-separated variable names
  label/.estore in               = \KLabel,     % label for the diagram
  label scale/.estore in         = \KLabelscale,% scale factor for values
  value scale/.estore in         = \KValscale,  % scale factor for values
  variable scale/.estore in      = \KVarscale,  % scale factor for variables
  variable sep/.estore in        = \KVarsep,    % inner sep for variables
  address scale/.estore in       = \KAddrscale, % scale factor for adresses
  address sep/.estore in         = \KAddrsep,   % inner sep for adresses
  group distance/.estore in      = \KGdist,     % distance for grouping lines
  group color/.estore in         = \KGcolor,    % grouping color
  group linewidth/.estore in     = \KGwidth,    % grouping line width
  group rounded/.estore in       = \KGrounded,  % grouping rounded corner radius
  group opacity/.estore in       = \KGopacity,  % grouping field opacity
  indicator distance/.estore in  = \KInddist,   % variable indicator distance
  indicator linewidth/.estore in = \KIndwidth,  % line width for indicators
  indicator scale/.estore in     = \KIndscale,  % scale factor for indicators
  indicator sep/.estore in       = \KIndsep,    % inner sep for indicators
  plot addresses/.is if=Kaddress,               % plot address fields
  plot indicators/.is if=KInd,                  % plot variable indicators
}
\pgfkeys{
  /K,
  default/.style = { plot addresses      = true, 
                     plot indicators     = true,
                     x bits              = 2, 
                     y bits              = 2,
                     address scale       = 0.5,
                     address sep         = 1pt,
                     variable names      = {$A$,$B$,$C$,$D$,$E$,$F$,$G$}, 
                     label               = $X$,
                     label scale         = 1.4, 
                     variable scale      = 0.7,
                     variable sep        = 2pt,
                     value scale         = 2,
                     group color         = red,
                     group rounded       = 2pt,
                     group linewidth     = .3mm,
                     group opacity       = 0.10,
                     group distance      = 0.4,
                     indicator distance  = 0.45,
                     indicator linewidth = .4mm,
                     indicator scale     = 1.0,
                     indicator sep       = 2pt,
                    }
}
\tikzset{
  Kcorners/.style = { rounded corners = \KGrounded },
     Kline/.style = { Kcorners, draw = \KGcolor, line width = \KGwidth },
      KInd/.style = { draw = black, line width = \KIndwidth },
     Kfill/.style = { Kcorners, fill = \KGcolor, opacity = \KGopacity },
     Krect/.style = { Kcorners, Kfill },
     right/.style = { black, anchor = west }, % nodes at right side
      left/.style = { black, anchor = east }, % nodes at left side
}
\let\ol\overline% just to abbreviate

%==============================================================================
% K[options]{table}
%
% takes the address,value pairs from the comma-separated list {table} and
% draws up the corresponding K diagram.
% the table does not need to be sorted, nor to be complete
%
\newcommand{\K}[2][]
{%
  \pgfkeys{/K, default, #1}%
  \def\newstr{#2}
  \tikzmath
  {
    function fromgray(\gcode,\bits) %
    {
      int \x, \pos, \leftbit, \twop;
      \leftbit = 0;
      \pos = 0;
      for \x in {0,...,\bits-1}%
      {
        \twop = 2^(\bits - \x -1);
        if \gcode >= \twop then %
        { % gray code is 1
          if \leftbit == 0 then %
          {
            \pos = \pos + \twop;
            \leftbit = 1;
          } else
          {
            \leftbit = 0;
          };
          \gcode = \gcode - \twop;
        } else
        { % gray code is 0
          if \leftbit == 0 then %
          {
            \pos = \pos;
          } else
          {
            \pos = \pos + \twop;
            \leftbit = 1;
          };
        };
      };
      return \pos;
    };
    %
    int \xbits, \ybits, \fields, \vars;
    \xbits  = \KXvars;
    \ybits  = \KYvars;
    \vars   = \xbits+\ybits;
    \fields = 2^(\vars);
  }
  \StrCut[\xbits]{\KVars}{,}{\xlabels}{\ylabels}
  \StrBefore[\ybits]{\ylabels}{,}[\ylabels]
  \StrSubstitute[0]{\xlabels}{,}{}[\xlabels]
  \StrSubstitute[0]{\ylabels}{,}{}[\ylabels]
  \path ($(0,0.5)+\KVarscale*(0,2ex)$) node[anchor = south,
    inner sep = \KVarsep, scale = \KVarscale] {\ttfamily\xlabels};
  \path (-0.5,0.5) node[anchor = north east,
    inner sep = \KVarsep, scale = \KVarscale] {\ttfamily\ylabels};
  \foreach [remember = \newstr as \workstr (initially \newstr)]
    \i in {1,...,\fields} {%
    \StrCut[2]{\workstr}{,}{\nowstr}{\newstr}
    \StrLen{\newstr}[\len]
    \ifthenelse{\equal{\len}{0}}{\def\newstr{,,}}{}%
    \StrLen{\nowstr}[\len]
    \ifthenelse{\equal{\len}{1}}{}%
    {%
      \StrCut[1]{\nowstr}{,}{\addrstr}{\valuestr}
      \tikzmath{
        int \xpos, \ypos, \xaddr, \yaddr, \addr;
        \addr = 0b\addrstr;
        \xaddr = int(\addr / 2^\ybits);
        \yaddr = \addr - \xaddr * 2^\ybits;
        \xpos = fromgray(\xaddr,\xbits);
        \ypos = fromgray(\yaddr,\ybits);
      } % end tikzmath
      % place field value
      \path ($(\xpos,-\ypos)$) node[anchor=center,
        inner sep = 0pt, scale = \KValscale](F\addrstr){\valuestr};
    }%
  }
  \tikzmath
  {
    int \x, \y, \xmax, \ymax, \value;
    \xmax = 2^\xbits - 1;
    \ymax = 2^\ybits - 1;
    { \draw ($(-0.5,0.5)$) -- ($(-0.5,-\ymax-0.5)$); };
    for \x in {0,...,\xmax}%
    {
      \xpos = fromgray(\x,\xbits);
      {
        \path ($(\xpos,0.5)$) node[anchor = south, black, inner sep = 2pt,
          scale = \KVarscale]{\ttfamily\padzeroes[\xbits]\binarynum{\x}};
        \draw ($(\xpos+0.5,0.5)$) -- ($(\xpos+0.5,-\ymax-0.5)$);
      };
    };
    { \draw ($(-0.5,+0.5)$) -- ($(\xmax+0.5,+0.5)$); };
    for \y in {0,...,\ymax}%
    {
      \ypos = fromgray(\y,\ybits);
      {
        \path ($(-0.5,-\ypos)$) node[anchor=east, black, inner sep = 2pt,
          scale = \KVarscale] {\ttfamily\padzeroes[\ybits]\binarynum{\y}};
        \draw ($(-0.5,-\ypos-0.5)$) -- ($(\xmax+0.5,-\ypos-0.5)$);
      };
    };
  }
  \ifKaddress
  \tikzmath
  {
    int \x, \y, \xmax, \ymax, \value;
    \xmax = 2^\xbits - 1;
    \ymax = 2^\ybits - 1;
    for \x in {0,...,\xmax}%
    {
      \xpos = fromgray(\x,\xbits);
      for \y in {0,...,\ymax}%
      {
        \ypos = fromgray(\y,\ybits);
        { %
            \path ($(\xpos,-\ypos)+(0.5,-0.5)$) node[anchor = south east,
              inner sep = \KAddrsep, blue, scale = \KAddrscale]
              {\ttfamily\padzeroes[\xbits]%
               \binarynum{\x}\padzeroes[\ybits]\binarynum{\y}};
        };
      };
    };
  }
  \fi
  \ifKInd
  \tikzmath%
  {%
    int \i, \il, \j, \n;
    real \x,\xmax,\y,\ymax;
    for \i in {1,...,\xbits}%
    {
      \n = 2^(\i-2);
      \il = \i+1;  
      if \n < 1 then {\n=1;};
      \y = 0.5 + \KInddist*(\i);
      for \j in {1,...,\n}%
      {
        \x = -0.5 + 2^(\xbits-\i) + (\j-1)*2^(\xbits-\i+2);
        \xmax = \x + 2^(\xbits-\i+1);
        if \xmax > (2^(\xbits)-0.5) then {\xmax=2^(\xbits)-0.5;};
        {
          \StrBetween[\i,\numexpr\i+1]{,\KVars,}{,}{,}[\tlabel] 
          \path [KInd] (\x,\y) -- (\xmax,\y) node[anchor = south east,
               inner sep = \KIndsep, scale = \KIndscale] {\tlabel};
          \path [KInd] (\x,\y)++(0,-0.1)    -- ++(0,+0.2);
          \path [KInd] (\xmax,\y)++(0,-0.1) -- ++(0,+0.2);
        };
      };
    };
    for \i in {1,...,\ybits}%
    {
      \n = 2^(\i-2);
      if \n < 1 then {\n=1;};
      \x = -0.5 - \KInddist*(\i-1);
      for \j in {1,...,\n}%
      {
        \y = -0.5 + 2^(\ybits-\i) + (\j-1)*2^(\ybits-\i+2);
        \ymax = \y + 2^(\ybits-\i+1);
        if \ymax > (2^(\ybits)-0.5) then {\ymax=2^(\ybits)-0.5;};
        {
          \StrBetween[\numexpr\i+\xbits,\numexpr\i+\xbits+1]
            {,\KVars,}{,}{,}[\tlabel]
          \path [KInd] ($(\x,-\y)-\ybits*(1.4ex,0)$)
            -- ($(\x,-\ymax)-\ybits*(1.4ex,0)$) node[anchor=south west, 
                rotate=90, inner sep=\KIndsep, scale=\KIndscale]{\tlabel};
          \path [KInd] ($(\x,-\y)-\ybits*(1.4ex,0)$)++(-0.1,0) -- ++(+0.2,0);
          \path [KInd] ($(\x,-\ymax)-\ybits*(1.4ex,0)$)++(-0.1,0) -- ++(+0.2,0);
        };
      };
    };
  }
  \fi
  \path[draw = black] (-0.5,0.5) -- ++(-0.5,0.5) node[anchor = south east,
    inner sep = 2pt, scale = \KLabelscale] {\KLabel};
}

%==============================================================================
% KG[options]{from}{to}
%
% groups together the fields between {from} and {to} by drawing a frame around 
% these fields.
% if the {from} field is above or right of the {to} field, the frame is
% extended around the edge of the diagram.
%
\newcommand{\KG}[3][] % options from to
{%
  \pgfkeys{/K, default, #1}%
  \tikzmath
  {
    int \xbits, \ybits, \fields, \vars;
    \xbits = \KXvars;
    \ybits = \KYvars;
    \vars  = \xbits+\ybits;
    \fields = 2^(\vars);
  }
  \tikzmath{
    int \xfrompos, \yfrompos, \xfrom, \yfrom, \fromaddr;
    int \xtopos, \ytopos, \xto, \yto, \toaddr;
    \fromaddr = 0b#2;
    \toaddr   = 0b#3;
    \xfrom    = int(\fromaddr / 2^\ybits);
    \yfrom    = \fromaddr - \xfrom * 2^\ybits;
    \xto      = int(\toaddr / 2^\ybits);
    \yto      = \toaddr - \xto * 2^\ybits;
    \xfrompos = fromgray(\xfrom,\xbits);
    \yfrompos = fromgray(\yfrom,\ybits);
    \xtopos   = fromgray(\xto,\xbits);
    \ytopos   = fromgray(\yto,\ybits);
    \xmax     = 2^\xbits-1;
    \ymax     = 2^\ybits-1;
  } % end tikzmath
  \tikzmath%
  {
    if (\xfrompos <= \xtopos) && (\yfrompos >= \ytopos) then %
    {
      { 
        \path[Kfill] ($(\xfrompos,-\yfrompos)-(\KGdist,\KGdist)$)
          rectangle ($(\xtopos,-\ytopos)+(\KGdist,\KGdist)$);
        \path[Kline] ($(\xfrompos,-\yfrompos)-(\KGdist,\KGdist)$)
          rectangle ($(\xtopos,-\ytopos)+(\KGdist,\KGdist)$);
      };
    };
    if (\xfrompos <= \xtopos) && (\yfrompos < \ytopos) then %
    {
      {
        \path[Krect] ($(\xfrompos,0)+(-\KGdist,+0.5)$)
          -- ($(\xfrompos,-\yfrompos)+(-\KGdist,-\KGdist)$)
          -- ($(\xtopos,-\yfrompos)+(\KGdist,-\KGdist)$)
          -- ($(\xtopos,0)+(\KGdist,+0.5)$) -- cycle;
        \path[Krect] ($(\xfrompos,-\ymax)+(-\KGdist,-0.5)$)
          -- ($(\xfrompos,-\ytopos)+(-\KGdist,+\KGdist)$)
          -- ($(\xtopos,-\ytopos)+(\KGdist,+\KGdist)$)
          -- ($(\xtopos,-\ymax)+(\KGdist,-0.5)$) -- cycle;
        \path[Kline] ($(\xfrompos,0)+(-\KGdist,+0.5)$)
          -- ($(\xfrompos,-\yfrompos)+(-\KGdist,-\KGdist)$)
          -- ($(\xtopos,-\yfrompos)+(\KGdist,-\KGdist)$)
          -- ($(\xtopos,0)+(\KGdist,+0.5)$);
        \path[Kline] ($(\xfrompos,-\ymax)+(-\KGdist,-0.5)$)
        -- ($(\xfrompos,-\ytopos)+(-\KGdist,+\KGdist)$)
        -- ($(\xtopos,-\ytopos)+(\KGdist,+\KGdist)$)
        -- ($(\xtopos,-\ymax)+(\KGdist,-0.5)$);
      };
    };
    if (\xfrompos > \xtopos) && (\yfrompos >= \ytopos) then %
    {
      {
        \path[Krect] ($(0,-\yfrompos)+(-0.5,-\KGdist)$)
          -- ($(\xtopos,-\yfrompos)+(+\KGdist,-\KGdist)$)
          -- ($(\xtopos,-\ytopos)+(\KGdist,+\KGdist)$)
          -- ($(0,-\ytopos)+(-0.5,+\KGdist)$) -- cycle;
        \path[Krect] ($(\xmax,-\yfrompos)+(+0.5,-\KGdist)$)
          -- ($(\xfrompos,-\yfrompos)+(-\KGdist,-\KGdist)$)
          -- ($(\xfrompos,-\ytopos)+(-\KGdist,+\KGdist)$)
          -- ($(\xmax,-\ytopos)+(+0.5,+\KGdist)$) -- cycle;
        \path[Kline] ($(0,-\yfrompos)+(-0.5,-\KGdist)$)
          -- ($(\xtopos,-\yfrompos)+(+\KGdist,-\KGdist)$)
          -- ($(\xtopos,-\ytopos)+(\KGdist,+\KGdist)$)
          -- ($(0,-\ytopos)+(-0.5,+\KGdist)$);
        \path[Kline] ($(\xmax,-\yfrompos)+(+0.5,-\KGdist)$)
          -- ($(\xfrompos,-\yfrompos)+(-\KGdist,-\KGdist)$)
          -- ($(\xfrompos,-\ytopos)+(-\KGdist,+\KGdist)$)
          -- ($(\xmax,-\ytopos)+(+0.5,+\KGdist)$);
      };
    };
    if (\xfrompos > \xtopos) && (\yfrompos < \ytopos) then %
    {
      {
        \path[Kfill] {[Kcorners] ($(0,-\yfrompos)+(-0.5,-\KGdist)$) 
          -- ($(\xtopos,-\yfrompos)+(+\KGdist,-\KGdist)$)
          -- ($(\xtopos,0)+(\KGdist,+0.5)$)}
          -- ($(0,0)+(-0.5,+0.5)$) -- cycle;
        \path[Kfill] {[Kcorners] ($(0,-\ytopos)+(-0.5,+\KGdist)$)
          -- ($(\xtopos,-\ytopos)+(+\KGdist,+\KGdist)$)
          -- ($(\xtopos,-\ymax)+(\KGdist,-0.5)$)}
          -- ($(0,-\ymax)+(-0.5,-0.5)$) -- cycle;
        \path[Kfill] {[Kcorners] ($(\xmax,-\yfrompos)+(+0.5,-\KGdist)$)
          -- ($(\xfrompos,-\yfrompos)+(-\KGdist,-\KGdist)$)
          -- ($(\xfrompos,0)+(-\KGdist,+0.5)$)}
          -- ($(\xmax,0)+(+0.5,+0.5)$) -- cycle;
        \path[Kfill] {[Kcorners] ($(\xmax,-\ytopos)+(+0.5,+\KGdist)$)
          -- ($(\xfrompos,-\ytopos)+(-\KGdist,+\KGdist)$)
          -- ($(\xfrompos,-\ymax)+(-\KGdist,-0.5)$)}
          -- ($(\xmax,-\ymax)+(+0.5,-0.5)$) -- cycle;
        \path[Kline] ($(0,-\yfrompos)+(-0.5,-\KGdist)$)
          -- ($(\xtopos,-\yfrompos)+(+\KGdist,-\KGdist)$)
          -- ($(\xtopos,0)+(\KGdist,+0.5)$);
        \path[Kline] ($(0,-\ytopos)+(-0.5,+\KGdist)$)
          -- ($(\xtopos,-\ytopos)+(+\KGdist,+\KGdist)$)
          -- ($(\xtopos,-\ymax)+(\KGdist,-0.5)$);
        \path[Kline] ($(\xmax,-\yfrompos)+(+0.5,-\KGdist)$)
          -- ($(\xfrompos,-\yfrompos)+(-\KGdist,-\KGdist)$)
          -- ($(\xfrompos,0)+(-\KGdist,+0.5)$);
        \path[Kline] ($(\xmax,-\ytopos)+(+0.5,+\KGdist)$)
          -- ($(\xfrompos,-\ytopos)+(-\KGdist,+\KGdist)$)
          -- ($(\xfrompos,-\ymax)+(-\KGdist,-0.5)$);
      };
    };
  }
  
}

\newcommand*\ul[2]{\tikz[baseline = (char.base)]{
  \node[inner sep = 2pt] (char) {#2};
  \draw[#1, line width = 2pt](char.south west) -- (char.south east);}}