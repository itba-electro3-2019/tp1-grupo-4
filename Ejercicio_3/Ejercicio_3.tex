\section{Ejercicio 3 - Implementación de módulos en verilog}
A continuación, se implementarán los circuitos pedidos en lenguaje verilog, comentando como fue su desarrollo e emplementación.
\subsection{ENCODER de 4 entradas}


\begin{table}[H]
	\begin{center}
		\begin{tabular}{|c|c|c||c|c|c|c|}
			\hline
			\multicolumn{3}{|c||}{Entrada} & \multicolumn{4}{|c|}{Salida}\\
			\hline
			$D$ &	$C_1$ &	$C_0$ &	$O_3$ & $O_2$ & $o_1$ &$O_0$ \\
			\hline
            0 & 0 & 0 & 0 & 0 & 0 & 0 \\
            \hline
            0 & 0 & 1 & 0 & 0 & 0 & 0 \\
            \hline
            0 & 1 & 0 & 0 & 0 & 0 & 0 \\
            \hline
            0 & 1 & 1 & 0 & 0 & 0 & 0 \\
            \hline
            1 & 0 & 0 & 1 & 0 & 0 & 0 \\
            \hline
            1 & 0 & 1 & 0 & 1 & 0 & 0 \\
            \hline
            1 & 1 & 0 & 0 & 0 & 1 & 0 \\
            \hline
            1 & 1 & 1 & 0 & 0 & 0 & 1\\
			\hline
		\end{tabular}
	\end{center}
\end{table}

\subsection{DEMUX de 4 salidas}

\begin{table}[H]
	\begin{center}
		\begin{tabular}{|c|c|c|c||c|c|}
			\hline
			\multicolumn{4}{|c||}{Entrada} & \multicolumn{2}{|c|}{Salida}\\
			\hline
			$D$ &	$C$ &	$B$ &	$A$ & $S_1$ & $S_0$ \\
			\hline
            0 & 0 & 0 & 0 & 0 & 0   \\
            \hline
            0 & 0 & 0 & 1 & 0 & 0   \\
            \hline
            0 & 0 & 1 & 0 & 0 & 1   \\
            \hline
            0 & 0 & 1 & 1 & 0 & 0   \\
            \hline
            0 & 1 & 0 & 0 & 1 & 0   \\
            \hline
            0 & 1 & 0 & 1 & 0 & 0   \\
            \hline
            0 & 1 & 1 & 0 & 0 & 0   \\
            \hline
            0 & 1 & 1 & 1 & 0 & 0   \\
            \hline
            1 & 0 & 0 & 0 & 1 & 1   \\
            \hline
            1 & 0 & 0 & 1 & 0 & 0   \\
            \hline
            1 & 0 & 1 & 0 & 0 & 0   \\
            \hline
            1 & 0 & 1 & 1 & 0 & 0  \\
            \hline
            1 & 1 & 0 & 0 & 0 & 0   \\
            \hline
            1 & 1 & 0 & 1 & 0 & 0   \\
            \hline
            1 & 1 & 1 & 0 & 0 & 0   \\
            \hline
            1 & 1 & 1 & 1 & 0 & 0  \\
			\hline
		\end{tabular}
	\end{center}
\end{table}