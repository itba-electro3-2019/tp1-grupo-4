\section{Algebra booleana y compuertas lógicas}

En el álgebra booleana se conoce como término canónico de una función lógica a todo producto o suma en la cual aparecen todas las variables en su forma directa o inversa. Cualquier función lógica puede expresarse de forma canónica utilizando los conceptos de min y maxtérminos. El primero, se refiere a todas las filas para las cuales la función lógica es igual a uno en la tabla de verdad  correspondiente, mientras que el segundo se corresponde con todos los valores para los cuales la función es igual a cero. 

En el siguiente ejercicio se simplificarán dos funciones lógicas aplicando ambos conceptos. Luego se utilizarán mapas de Karnaugh para llegar a la misma expresión.A continuación se dibujará el circuito lógico resultante utilizando compuertas AND, OR y NOT, y por último, se usarán compuertas NAND exclusivamente.

\subsection{Primera Expresión: suma de productos}

\subsubsection{Simplificación mediante álgebra booleana}

La expresión de la cual se parte está dada por la siguiente fórmula:

\begin{equation}\label{eq:suma_minterminos}
f(e,d,c,b,a) = \sum{m(0, 2, 4, 7, 8, 10, 12, 16, 18, 20, 23, 24, 25, 26, 27, 28)}
\end{equation}

La ecuación \ref{eq:suma_minterminos} nos indica cuales entradas en la tabla de verdad son iguales a uno. A partir de ella podemos plantear la siguiente ecuación, la cual se trabajará para llegar a su respectiva expresión canónica.

\begin{multline}
\bar{e} \cdot \bar{d} \cdot \bar{c} \cdot \bar{b} \cdot \bar{a} + 
\bar{e} \cdot \bar{d} \cdot \bar{c} \cdot b \cdot \bar{a} +
\bar{e} 	\cdot \bar{d} \cdot c \cdot \bar{b} \cdot \bar{a} +
\bar{e} \cdot \bar{d} \cdot c \cdot b \cdot a + 
\bar{e} \cdot d \cdot \bar{c} \cdot \bar{b} \cdot \bar{a} + 
\bar{e} \cdot d \cdot \bar{c} \cdot b \cdot \bar{a} + 
\bar{e} \cdot d \cdot c \cdot \bar{b} \cdot \bar{a} + 
e \cdot \bar{d} \cdot \bar{c} \cdot \bar{b} \cdot \bar{a} + \\
e \cdot \bar{d} \cdot \bar{c} \cdot b \cdot \bar{a} + 
e \cdot \bar{d} \cdot c \cdot \bar{b} \cdot \bar{a} + 		
e \cdot \bar{d} \cdot c \cdot b \cdot a + 
e \cdot d \cdot \bar{c} \cdot \bar{b} \cdot \bar{a} + 
e \cdot d \cdot \bar{c} \cdot \bar{b} \cdot a + 
e \cdot d \cdot \bar{c} \cdot b \cdot \bar{a} + 
e \cdot d \cdot \bar{c} \cdot b \cdot a +
e \cdot d \cdot c \cdot \bar{b} \cdot \bar{a} 
\end{multline}

La expresión anterior puede parecer complicada en un principio, pero esta ordenada según cada mintermino correspondiente. Para comenzar el proceso de simplfiicación, gracias a la ley de idempotencia se puede sumar términos ya presentes en la ecuación anterior. So procederá a duplicar los mintérminos 2, 18 y 26 y a ordenar la fórmula para faciliatr la simplificación. 

\begin{equation}
\bar{e} \cdot \bar{d} \cdot \bar{c} \cdot \bar{b} \cdot \bar{a} +  %0
\bar{e} \cdot \bar{d} \cdot \bar{c} \cdot b \cdot \bar{a} +        %2
e \cdot \bar{d} \cdot \bar{c} \cdot \bar{b} \cdot \bar{a} +    	  %16
e \cdot \bar{d} \cdot \bar{c} \cdot b \cdot \bar{a}  	          %18
\end{equation}

\begin{equation}
\bar{e} \cdot d \cdot \bar{c} \cdot b \cdot \bar{a} + 		      %10
e \cdot d \cdot \bar{c} \cdot b \cdot \bar{a} + 	                  %26
\bar{e} \cdot \bar{d} \cdot \bar{c} \cdot b \cdot \bar{a} +        %2
e \cdot \bar{d} \cdot \bar{c} \cdot b \cdot \bar{a}  	          %18
\end{equation}

\begin{equation}
\bar{e} 	\cdot \bar{d} \cdot c \cdot \bar{b} \cdot \bar{a} +	      %4
\bar{e} \cdot d \cdot c \cdot \bar{b} \cdot \bar{a} + 			  %12
e \cdot \bar{d} \cdot c \cdot \bar{b} \cdot \bar{a} + 		      %20
e \cdot d \cdot c \cdot \bar{b} \cdot \bar{a}                      %28
\end{equation}

\begin{equation}
e \cdot d \cdot \bar{c} \cdot \bar{b} \cdot \bar{a} +              %24
e \cdot d \cdot \bar{c} \cdot \bar{b} \cdot a +                    %25
e \cdot d \cdot \bar{c} \cdot b \cdot \bar{a} + 	                  %26
e \cdot d \cdot \bar{c} \cdot b \cdot a                           %27
\end{equation}

\begin{equation}
e \cdot \bar{d} \cdot c \cdot b \cdot a +                          %23
\bar{e} \cdot \bar{d} \cdot c \cdot b \cdot a  			          %7
\end{equation}

Las sumas de productos anteriores pueden trabajarse si se factorizan por los términos apropiados:

\begin{equation}\label{eq:minterm_0_2_16_18}
\bar{d} \cdot \bar{c} \cdot \bar{a} \cdot (\bar{e} \cdot \bar{b} + \bar{e} \cdot b + e \cdot \bar{b} + e \cdot b)
\end{equation}

\begin{equation}\label{eq:mintem_10_26_2_18}
\bar{c} \cdot b \cdot \bar{a} \cdot (\bar{e} \cdot b + e 	\cdot d + \bar{e} \cdot \bar{d} + e \cdot \bar{d})
\end{equation}

\begin{equation}\label{eq:minterm_4_12_20_28}
c \cdot \bar{b} \cdot \bar{a} \cdot (\bar{e} \cdot \bar{d} + \bar{e} \cdot d + e \cdot \bar{d} + e \cdot d)
\end{equation}

\begin{equation}\label{eq:minterm_24_25_26_27}
e \cdot d \cdot c \cdot (\bar{b} \cdot \bar{a} + \bar{b} \cdot a + b \cdot \bar{a} + b \cdot a)
\end{equation}

\begin{equation}\label{eq:minterm_/_23}
\bar{d} \cdot c \cdot b \cdot a \cdot (\bar{e} + e)
\end{equation}

Luego aplicando la propiedad de combinacion en todos los productos:

\begin{equation}\label{eq:minterm_0_2_16_18}
\bar{d} \cdot \bar{c} \cdot \bar{a} \cdot (\bar{e} \cdot \bar{b} + \bar{e} \cdot b + e \cdot \bar{b} + e \cdot b)
\end{equation}

\begin{equation}\label{eq:mintem_10_26_2_18}
\bar{c} \cdot b \cdot \bar{a} \cdot (\bar{e} \cdot b + e 	\cdot d + \bar{e} \cdot \bar{d} + e \cdot \bar{d})
\end{equation}

\begin{equation}\label{eq:minterm_4_12_20_28}
c \cdot \bar{b} \cdot \bar{a} \cdot (\bar{e} \cdot \bar{d} + \bar{e} \cdot d + e \cdot \bar{d} + e \cdot d)
\end{equation}

\begin{equation}\label{eq:minterm_24_25_26_27}
e \cdot d \cdot c \cdot (\bar{b} \cdot \bar{a} + \bar{b} \cdot a + b \cdot \bar{a} + b \cdot a)
\end{equation}

\begin{equation}\label{eq:minterm_7_23}
\bar{d} \cdot c \cdot b \cdot a \cdot (\bar{e} + e)
\end{equation}


En \ref{eq:minterm_0_2_16_18}, \ref{eq:mintem_10_26_2_18}, \ref{eq:minterm_4_12_20_28} y \ref{eq:minterm_24_25_26_27} se puede aplicar la propiedad de combinación, mientras que en \ref{eq:minterm_/_23} se utiliza la ley del complemento:

\begin{equation}\label{eq:minterm_0_2_16_18_Simp}
\bar{d} \cdot \bar{c} \cdot \bar{a} \cdot (\bar{e} + e)
\end{equation}

\begin{equation}\label{eq:mintem_10_26_2_18_Simp}
\bar{c} \cdot b \cdot \bar{a} \cdot (d + \bar{d})
\end{equation}

\begin{equation}\label{eq:minterm_4_12_20_28_Simp}
c \cdot \bar{b} \cdot \bar{a} \cdot (\bar{e} + e)
\end{equation}

\begin{equation}\label{eq:minterm_24_25_26_27_Simp}
e \cdot d \cdot c \cdot (\bar{b} + b)
\end{equation}

\begin{equation}\label{eq:minterm_7_23_Simp}
\bar{d} \cdot c \cdot b \cdot a 
\end{equation}

Sumando \ref{eq:minterm_0_2_16_18_Simp}, \ref{eq:mintem_10_26_2_18_Simp}, \ref{eq:minterm_4_12_20_28_Simp}, \ref{eq:minterm_24_25_26_27_Simp} y \ref{eq:minterm_7_23_Simp}:

\begin{equation}
\bar{d} \cdot \bar{c} \cdot \bar{a} \cdot (\bar{e} + e) + 
\bar{c} \cdot b \cdot \bar{a} \cdot (d + \bar{d}) + 
c \cdot \bar{b} \cdot \bar{a} \cdot (\bar{e} + e) + 
e \cdot d \cdot c \cdot (\bar{b} + b) +
\bar{d} \cdot c \cdot b \cdot a 
\end{equation}

Finalmente se aplica nuevamente la ley del complemento en los cuatro primeros productos y se llega a la expresión final:

\begin{equation}\label{canonica_minterm}
\boxed{\bar{d} \cdot \bar{c} \cdot \bar{a} + 
\bar{c} \cdot b \cdot \bar{a} + 
c \cdot \bar{b} \cdot \bar{a} + 
e \cdot d \cdot c \cdot + 
\bar{d} \cdot c \cdot b \cdot a}
\end{equation}


\subsubsection{Simplificacioón mediante mapas de Karnaugh}
    \begin{tikzpicture}[x=1cm,y=1cm]
        \K[x bits = 2, y bits = 2, label={$$},
        variable names = {$D$,$C$,$B$,$A$,}]
        { 
        0000,1,    
        0001,0,   
        0010,1,   
        0011,0,       
        0100,1, 
        0101,0,
        0110,0,
        0111,0,
        1000,1,    
        1001,0,   
        1010,1,   
        1011,0,       
        1100,1, 
        1101,0,
        1110,0,
        1111,1,
        }
        \newcommand*{\myKG}[4][0.1]{\KG[x bits = 2,y bits = 2,group opacity = #1,
                    #2]{#3}{#4}}
        \myKG     {group color = brown,  group distance=0.35}{0000}{1000}
        \myKG     {group color = green,  group distance=0.35}{0010}{0010}
        \myKG     {group color = red,  group distance=0.35}{1010}{1010}
        \path (1,-3.5) node[anchor = north, align = left] (eq1){%
        $
        \ul{brown}{$\ol{B}\,\ol{A}\,$} +
        \ul{brown}{$\ol{D}\,\ol{C}\,B\,\ol{A}\,$}
        $};
    \end{tikzpicture}
    
    \begin{tikzpicture}[x=1cm,y=1cm]
        \K[x bits = 2, y bits = 2, label={$$},
        variable names = {$E$,$D$,$C$,$B$,}]
        { 
        0000,1,    
        0001,1,   
        0010,1,   
        0011,0,       
        0100,1, 
        0101,1,
        0110,1,
        0111,0,
        1000,1,    
        1001,1,   
        1010,1,   
        1011,0,       
        1100,1, 
        1101,1,
        1110,1,
        1111,0,
        }
        \newcommand*{\myKG}[4][0.1]{\KG[x bits = 2,y bits = 2,group opacity = #1,
                    #2]{#3}{#4}}
        \myKG     {group color = brown,  group distance=0.35}{0001}{1000}
        \myKG     {group color = green,  group distance=0.35}{0000}{1010}
        \path (1,-3.5) node[anchor = north, align = left] (eq1){%
        $A=0 /                 
        \ul{green}{$\ol{B}\,$} +
        \ul{brown}{$\ol{C}\,$}
        $};
    \end{tikzpicture}
    
    \begin{tikzpicture}[x=1cm,y=1cm]
        \K[x bits = 2, y bits = 2, label={$$},
        variable names = {$E$,$D$,$C$,$B$,}]
        { 
        0000,0,    
        0001,0,   
        0010,0,   
        0011,1,       
        0100,0, 
        0101,0,
        0110,0,
        0111,0,
        1000,0,    
        1001,0,   
        1010,0,   
        1011,1,       
        1100,1, 
        1101,1,
        1110,0,
        1111,0,
        }
        \newcommand*{\myKG}[4][0.1]{\KG[x bits = 2,y bits = 2,group opacity = #1,
                    #2]{#3}{#4}}
        \myKG     {group color = brown,  group distance=0.35}{1101}{1100}
        \myKG     {group color = green,  group distance=0.35}{1011}{0011}
        \path (1,-3.5) node[anchor = north, align = left] (eq1){%
        $A=1 /                 
        \ul{green}{$C\,B\,\ol{D}\,$} +
        \ul{brown}{$E\,D\,\ol{C}\,$}
        $};
    \end{tikzpicture}
    
    \pagebreak

    \begin{table}[H]
        \begin{center}
            \def\arraystretch{1.5}
            \begin{tabular}{|c|c|}
                \hline
                (0,2,8,10,16,18,24,26) &	$\bar{C}\bar{E}$ \\
                \hline
                (0,4,8,12,16,20,24,28) & $\bar{D}\bar{E}$ \\
                \hline  
                (24,25,26,27) & $AB\bar{C}$ \\
                \hline   
                (7,23) & $\bar{B}CDE$ \\
                \hline           
                
            \end{tabular}
            %\caption{Tabla de verdad del Demultiplexor}
        \end{center}

    \end{table}

    \pagebreak
    
    \begin{table}[H]
        \begin{center}
            \def\arraystretch{1.5}
            \begin{tabular}{|c|c|c|c|c|}
                \hline
                &	$\bar{D}\bar{E}$ &	$\bar{D}E$ &	$D\bar{E}$ & $DE$ \\
                \hline
                $\bar{A}\bar{B}\bar{C}$ & 1 & 0 & 0 & 1 \\
                \hline          
                $\bar{A}\bar{B}C$ & 1 & 0 & 1 & 0 \\
                \hline   
                $\bar{A}BC$ & 1 & 0 & 0 & 0 \\
                \hline     
                $\bar{A}B\bar{C}$  & 1 & 0 & 0 & 1 \\
                \hline
                
                $A\bar{B}\bar{C}$ & 1 & 0 & 0 & 1 \\
                
                \hline
                
                $A\bar{B}C$ & 1 & 0 & 1 & 0 \\
                \hline
                
                $ABC$ & 1 & 0 & 0 & 0 \\
                \hline
                
                $AB\bar{C}$ & 1 & 1 & 1 & 1 \\
                \hline
                
            \end{tabular}
            %\caption{Tabla de verdad del Demultiplexor}
        \end{center}

    \end{table}

    \pagebreak

    \begin{table}[H]
        \begin{center}
            \def\arraystretch{1.5}
            \begin{tabular}{|c|c|c|c|c|}
                \hline
                &	$\bar{D}\bar{E}$ &	$\bar{D}E$ &	$D\bar{E}$ & $DE$ \\
                \hline
                $\bar{A}\bar{B}\bar{C}$ & 0 & 1 & 3 & 2 \\
                \hline          
                $\bar{A}\bar{B}C$ & 4 & 5 & 7 & 6 \\
                \hline   
                $\bar{A}BC$ & 12 & 13 & 15 & 14 \\
                \hline     
                $\bar{A}B\bar{C}$  & 8 & 9 & 11 & 10 \\
                \hline
                
                $A\bar{B}\bar{C}$ & 16 & 17 & 19 & 18 \\
                
                \hline
                
                $A\bar{B}C$ & 20 & 21 & 23 & 22 \\
                \hline
                
                $ABC$ & 28 & 29 & 31 & 30 \\
                \hline
                
                $AB\bar{C}$ & 24 & 25 & 27 & 26 \\
                \hline
                
            \end{tabular}
            %\caption{Tabla de verdad del Demultiplexor}
        \end{center}

    \end{table}


\subsubsection{Implementación mediante compuertas AND, OR y NOT}

La expresión obtenida en \ref{canonica_minterm} puede ser implementada mediante compuertas lógicas facilmente:

\includegraphics{Ejercicio_2/circuitos/Ej2_parte1_logly.png}


\subsubsection{Implementación mediante compuertas NAND}
A su vez el circuito anterior implementado mediante compuertas nand queda de la siguiente manera:

\includegraphics{Ejercicio_2/circuitos/Ej2_parte1_nand_logly.png}



\subsection{Segunda expresión: producto de sumas}

\subsubsection{Simplificación mediante álgebra booleana}
Para la segunda parte del ejercicio, se comienza a partir del siguiente producto de sumas:
\begin{equation}\label{eq:prod_maxterminos}
f(d,c,b,a) = \prod{M_0, M_2, M_4, M_7, M_8, M_10, M_12)}
\end{equation}

A a partir de \ref{eq:prod_maxterminos} se define la siguiente expresion:

\begin{equation}
(d + c + b +a) \cdot (d + c + \bar{b} + a) \cdot (d + \bar{c} +b +a) \cdot (d + \bar{c} + \bar{b} + \bar{a}) 	\cdot (\bar{d} + c + b + a) \cdot (\bar{d} + c + \bar{b} + a) \cdot (\bar{d} + \bar{c} + b + a)
\end{equation}

En este caso se usa nuevamente la propiedad de idempotencia, en este caso para un producto, para duplicar el mintermino número 8 y 0 y así facilitar el trabajo algebraico. Luego se separan los productos en tres grupos para su posterior factorización:

\begin{equation}\label{Maxterm_0_2_8_10}
(d + c + b +a) \cdot (d + c + \bar{b} + a) \cdot (\bar{d} + c + b + a) \cdot (\bar{d} + c + \bar{b} + a)
\end{equation}

\begin{equation}\label{Maxterm_0_4_8_12}
(d + c + b +a) \cdot (d + \bar{c} +b +a) \cdot (\bar{d} + c + b + a) \cdot (\bar{d} + \bar{c} + b + a)
\end{equation}

\begin{equation}\label{Maxterm_7}
(d + \bar{c} + \bar{b} + \bar{a})
\end{equation}

Tanto en \ref{Maxterm_0_2_8_10} como en \ref{Maxterm_0_4_8_12} es posible usar la propiedad de combianción para el producto entre los primeros dos y los últimos dos términos para simplificar las expresiones. \ref{Maxterm_7} esta expresado en forma canóncia.

\begin{equation}\label{Maxterm_0_2_8_10_simp}
(d + c + a) \cdot (\bar{d} + c + a) 
\end{equation}

\begin{equation}\label{Maxterm_0_4_8_12_simp}
(d + b +a) \cdot (\bar{d} + b + a)
\end{equation}

Luego se utiliza la propiedad de combianción para el producto nuevamente, en ambas ecuaciones:

\begin{equation}\label{Maxterm_0_2_8_10_final}
(c + a) 
\end{equation}

\begin{equation}\label{Maxterm_0_4_8_12_final}
(b +a)
\end{equation}

Por último se unen \ref{Maxterm_0_2_8_10_final}, \ref{Maxterm_0_4_8_12_final} y \ref{Maxterm_7} para llegar a una expresión final:

\begin{equation}\label{eq:max_final}
\boxed{(c + a) \cdot (b +a) \cdot (d + \bar{c} + \bar{b} + \bar{a})}
\end{equation}

\subsubsection{Simplificacioón mediante mapas de Karnaugh}

%Agregar Karnaugh

\subsubsection{Implementación mediante compuertas AND, OR y NOT}

Al igual que con la expresion anterior se implementó la función obtenida en \ref{eq:max_final} mediante compuertas AND, OR y NOT:

\includegraphics{Ejercicio_2/circuitos/Ej2_parte2_logly.png}

\subsubsection{Implementación mediante compuertas NAND}
Por último se realizó el circuito anterior solamente con compuertas nand:

\includegraphics{Ejercicio_2/circuitos/Ej2_parte2_nand_logly.png}